\documentclass[10pt,a4paper]{article}
\usepackage[latin1]{inputenc}
\usepackage{listings}
\usepackage{amsmath}
\usepackage{amsfonts}
\usepackage{amssymb}
\usepackage{txfonts}
\usepackage{graphicx}
\author{Alex Foglia\\ Tommaso Puccetti}
\title{Multivariate Analysis and Statistical Learning \\ Relazione Contest \\ Implementazione PC Algorithm}
\begin{document}
	\maketitle
	\tableofcontents
	\newpage
	\section{Accenni di Teoria}
	Un esempio di modello grafico sono le reti bayesiane. Tali reti sono rappresentate attraverso l'utilizzo di un Grafo Aciclico Diretto, o DAG.\\*
	Un DAG � un grafo diretto in cui non compaiono cicli, dove per ciclo si intende un qualunque cammino finito che, a partire da un nodo iniziale $v$ termini in $v$. Una rete bayesiana, rappresentata attraverso un DAG, ha delle applicazioni interessanti in contesti di Machine Learning e in particolare di analisi causale.\\*
	Sia $G = (V,E)$ un DAG su un insieme finito $X = \{X_v \forall v \in V\}$ di variabili casuali, allora:
	$$
	\forall u,v \in V \;non\;adiacenti\;| v \in nd(u) \Rightarrow u \Perp v | nd(u) - v
	$$
	Dove $nd(u)$ � l'insieme dei nodi \emph{non discendenti} di u, ossia tutti quei nodi $u'$ per cui non esiste un cammino da $u$ a $u'$.\\
	Dato un insieme di variabili osservate, con distribuzione di probabilit� congiunta gaussiana, � possibile imparare il DAG sottostante al campione osservato. A questo scopo � stato progettato un particolare algoritmo: il PC-Algorithm.\\*
	Esso � composto da due sotto-funzioni che risolvono due diversi problemi:
	\begin{itemize}
		\item La costruzione dello scheletro (o grafo morale)
		\item La costruzione del DAG a partire da un dato scheletro
	\end{itemize}
	\subsection{Costruzione dello scheletro}
	La prima fase non produce esclusivamente lo scheletro, infatti in essa computiamo anche il separation set ossia uninsieme di variabili associato a ciascuna coppia di variabili indipendenti $x,y$. Gli elementi di tale insieme rappresentano tutte quelle variabili che condizionano l'indipendenza fra $x$ e $y$, e che quindi si trovano nel cammino da $x$ a $y$. Lo pseudo-codice dell'algoritmo, che prende in ingresso la $z-$trasformata delle correlazioni parziali stimate e il $tuning\;parameter$ $\alpha$ � il seguente: 
\begin{lstlisting}
G = grafo_completo()
l = -1
repeat
  l = l + 1
  repeat
    seleziona una coppia ordinata di variabili adiacenti i,j in G
    se |adj(i,G)\{j}| >= l
      repeat TEST
        seleziona K tra i nodi adiacenti di i escluso j, con |K|=l
	    se sqrt(n-|K|-3)|Z(i,j|K)| <= phi_inverse(1-alpha/2)
	      cancella l'arco i,j da G
	      salva K nel separation set di [i][j] e di [j][i]
	      esci da TEST
    finch� tutti i K tali per cui |K| = l sono stati selezionati
  finch� tutte le coppie adiacenti sono state testate
finch� l > |adj(i,G)\{j}| per ogni i,j
\end{lstlisting}
\end{document}