\documentclass[xcolor ={table,usenames,dvipsnames}]{beamer}
\usepackage[italian]{babel}
\usepackage{color}
\usepackage{txfonts}
\PassOptionsToPackage{dvipsnames}{xcolor}
\title{Multivariate Analysis and Statistical Learning \\PC Algorithm's implementation}
\author{Authors: Alex Foglia, Tommaso Puccetti}
\institute{Universit\`a  degli Studi di Firenze}
\date{21/12/2018}
\usepackage{sansmathaccent}
\usetheme{Berlin} 
\useinnertheme{rounded}
\useoutertheme{miniframes} 
\setbeamercovered{dynamic}
\theoremstyle{definition}
\newtheorem{definizione}{Definizione}

\begin{document}
	
	\begin{frame}
		\maketitle
	\end{frame}

	\begin{frame}
		\frametitle{Theoretical references (1)}
		\begin{itemize}
			\item Bayesian Networks can be rappresented as a \textbf{directed acyclic graph (DAG)}
			\item "acyclic" means that there are no paths starting from a node $v$ that ends with $v$ itself, $\forall v \in G$
			
		\end{itemize}
	\end{frame}

	\begin{frame}
		\frametitle{Theoretical references (2)}
		Let $G = (V,E)$ be a DAG relative to a finite set  $X = \{X_v \forall v \in V\}$ of casual variables, then:
		$$
		\forall u,v \in V \;non\;adjacent\;| v \in nd(u) \Rightarrow u \Perp v | nd(u) - v
		$$
	Where $nd(u)$ is the set of \textbf{non-descendant} n of n, that are all those nodes $u'$ for which there is no path from $u$ to $u'$. \\
	\end{frame}

	\begin{frame}
		\frametitle{PC-Algorithm}
		Given a set of variables with a joint Gaussian probability distribution, it is possible to learn the DAG closer to the sample through the use of  \textbf{PC-Algorithm}. \\
		It is composed of two sub-functions that solve two different problems:
		\begin{enumerate}
			\item The construction of the skeleton (or \textbf{Moral Graph})
			\item The construction of the DAG from a given skeleton
		\end{enumerate}
	\end{frame}


	\begin{frame}
		\frametitle{Step one: read the dataset}
		\begin{itemize}
			\item import \textbf{pandas} library
			\item call \textbf{pandas.read$\_$csv()} function to read dataset
			\item define \textbf{alpha}
			\item call \textbf{get\_skeleton} on dataset and alpha as arguments
		\end{itemize}
	%	\begin{figure}[h!]
	%		\centering
	%		\includegraphics[scale=0.5]{img/dataset.PNG}
	%		\label{Interfacce di un CS}
	%	\end{figure}
		
	\end{frame}
	\begin{frame}
\frametitle{Step two: initialization}
\begin{itemize}
	\item read names of the dataset variables accessing \textbf{dataset.columns} field
	\item retrieve the correlation matrix of the given dataset with \textbf{dataset.corr().values}
	\item initialize \textbf{N,n} as the number of sampling and the number of variables
	\item initialize \textbf{G} as the complete graph of dimension n
	\item initalize the \textbf{separation\_set} as a list of list
	\item initialize \textbf{l = 0}, \textbf{stop = false}
\end{itemize}
%	\begin{figure}[h!]
%		\centering
%		\includegraphics[scale=0.5]{img/initialization.PNG}
%		\label{Interfacce di un CS}
%	\end{figure}

\end{frame}
\begin{frame}
\frametitle{Step three: define adj function}
\begin{itemize}
	\item define the \textbf{adj} function in order to get the adjacents of a node in a given graph
\end{itemize}
%	\begin{figure}[h!]
%		\centering
%		\includegraphics[scale=0.5]{img/adj.PNG}
%		\label{Interfacce di un CS}
%	\end{figure}

\end{frame}
\begin{frame}
\frametitle{Step four: how many variables are actually dependent?}
\begin{itemize}
	\item set stop condition to true
	\item retrieve dependent variables: i,j are actually dependent if the adjacence matrix[i][j] is equal to 1
	\item call the set of dependent variables \textbf{act\_dep}
\end{itemize}
%	\begin{figure}[h!]
%		\centering
%		\includegraphics[scale=0.5]{img/depvar.PNG}
%		\label{Interfacce di un CS}
%	\end{figure}

\end{frame}
\begin{frame}
\frametitle{Step five: variables needed for independence test}
\begin{itemize}
	\item for \textbf{x,y} in \textbf{act\_dep}
	\item retrieve the \textbf{neighbors} of \textbf{x} calling the \textbf{adj()} function
	\item remove y from the \textbf{neighbors} set
	\item if \textbf{neighbors} set has dimension $\ge$ \textbf{l} then \begin{itemize}
	\item if \textbf{neighbors} set has dimension $>$ \textbf{l} go ahead
	\end{itemize}
\end{itemize}
%	\begin{figure}[h!]
%		\centering
%		\includegraphics[scale=0.5]{img/indepvar.PNG}
%		\label{Interfacce di un CS}
%	\end{figure}
\end{frame}
\begin{frame}
\frametitle{Step six: conditional independence test}
\begin{itemize}
	\item foreach set \textbf{K} of neighbors of dimension \textbf{l}
	\item test independence of \textbf{x} and \textbf{y} given \textbf{K}
	\item if the p value is greater than \textbf{alpha}:
	\begin{itemize}
		\item remove the edge x,y setting \textbf{G[x][y] = 0}
		\item set \textbf{K} as the \textbf{separation\_set[x][y]}
	\end{itemize}
\end{itemize}
%	\begin{figure}[h!]
%		\centering
%		\includegraphics[scale=0.5]{img/indeptest.PNG}
%		\label{Interfacce di un CS}
%	\end{figure}
\end{frame}
\begin{frame}
\frametitle{Step eight: from the skeleton to the CPDAG}
\begin{itemize}
	\item return \textbf{G} and \textbf{separation\_set}
	\item call \textbf{to\_cpdag(G, separation\_set)}
\end{itemize}
%	\begin{figure}[h!]
%		\centering
%		\includegraphics[scale=0.5]{img/tocpdag.PNG}
%		\label{Interfacce di un CS}
%	\end{figure}
\end{frame}

\begin{frame}
\frametitle{Step eight: define the getIndependents() function}
\begin{itemize}
	\item define \textbf{getIndependents(adj\_matrix,reqij, reqji)}
	\item this function retrieve all the variables i,j such that: \textbf{adj\_matrix[i][j] == reqij} and \textbf{adj\_matrix[j][i] == reqji}
\end{itemize}
%	\begin{figure}[h!]
%		\centering
%		\includegraphics[scale=0.5]{img/getindep.PNG}
%		\label{Interfacce di un CS}
%	\end{figure}
\end{frame}

\begin{frame}
\frametitle{Step nine: CPDAG initialization}
\begin{itemize}
	\item set the \textbf{cpdag} as the skeleton
	\item set \textbf{dip} as the set of variables i,j for which exists an edge from i to j
\end{itemize}
%	\begin{figure}[h!]
%		\centering
%		\includegraphics[scale=0.5]{img/cpdaginit.PNG}
%		\label{Interfacce di un CS}
%	\end{figure}
\end{frame}
\begin{frame}
\frametitle{Step ten: rule "zero"}
\begin{itemize}
	\item foreach pair x,y in \textbf{dip}:
	\item add to \textbf{allZ} all the variables \textbf{z} for which exists an egde from \textbf{z} to \textbf{j} and \textbf{z is not x}
	\item if:\\\textbf{there is no edge between x and z}\\ \textbf{there is a separation set between x and z}\\\textbf{there is a separation set between z and x}\\\textbf{y is not in separation set between x and z} or \textbf{in separation set between z and x}, then:
	\item remove the edge from y to x and from z to y
\end{itemize}
%	\begin{figure}[h!]
%		\centering
%		\includegraphics[scale=0.5]{img/rulezero.PNG}
%		\label{Interfacce di un CS}
%	\end{figure}
\end{frame}
\begin{frame}
\frametitle{Step eleven: apply rules}
\begin{itemize}
	\item using the same logic we apply the known rules 1,2 and 3
	\item return the resulting cpdag
	\item using \textbf{matplotlib} and \textbf{networkx} we are able to plot the resulting cpdag
\end{itemize}
%	\begin{figure}[h!]
%		\centering
%		\includegraphics[scale=0.5]{img/rulezero.PNG}
%		\label{Interfacce di un CS}
%	\end{figure}
\end{frame}

\begin{frame}
\frametitle{Python vs R}
\begin{itemize}
	\item consider this two codes
\end{itemize}
%	\begin{figure}[h!]
%		\centering
%		\includegraphics[scale=0.5]{img/r.PNG}
%		\label{Interfacce di un CS}
%	\end{figure}
\end{frame}
	
\end{document}